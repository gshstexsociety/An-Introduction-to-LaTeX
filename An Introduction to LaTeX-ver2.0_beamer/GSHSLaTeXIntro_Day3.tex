\documentclass[12pt]{beamer}
\usetheme{Hannover}
\usepackage{graphicx}
\usepackage{booktabs}
\usepackage[english]{babel}
\usepackage{kotex}
%\usepackage[pdfencoding=auto]{hyperref}
\hypersetup{pdfencoding=auto}
\usepackage{ulem}
\usepackage[per-mode=symbol]{siunitx}
\sisetup{inter-unit-product =$\cdot$}
%\usepackage{verbatim}
\usepackage{listings}
%\usepackage{hyperref}
%\usepackage{url}
\lstset{language = [LaTeX]{TeX},
	basicstyle=\scriptsize\linespread{1}\ttfamily,
	tabsize=4,
	backgroundcolor=\color[gray]{0.9},
	showstringspaces=false
}
\usepackage{tikz}
\setbeamertemplate{caption}[numbered]
\graphicspath{{images/}}

\title[\LaTeX - Day 3]{\LaTeX 입문 - Day 3}

\author{경기과학고 \TeX 사용자협회}
\institute[GSHSTeXSociety]{\url{latex.gs.hs.kr}}
\date{마지막 수정일 : \today}

%\setbeamertemplate{navigation symbols}{}%to suppress navigation tools

\AtBeginSection[]{
	\begin{frame}
		\vfill
		\centering
		\begin{beamercolorbox}[sep=8pt,center,shadow=true,rounded=true]{title}
			\usebeamerfont{title}\insertsectionhead\par
		\end{beamercolorbox}
		\vfill
	\end{frame}
}

\newcommand{\MyBeginSection}[1]{
	\begin{frame}
		\vfill
		\centering
		\begin{beamercolorbox}[sep=8pt,center,shadow=true,rounded=true]{title}
			\usebeamerfont{title}\insertsectionhead\par
		\end{beamercolorbox}
		\footnotesize{#1}
		\vfill
	\end{frame}
}

\begin{document}

\begin{frame}
\titlepage % Print the title page as the first slide
\end{frame}

\begin{frame}{지난 시간에는}
	\begin{itemize}
		\item 라벨링 및 상호 참조
		\item 그림과 표의 삽입
		\item 논문 작성에 도움이 되는 팁들
	\end{itemize}
	\vfill
	이번 시간에는 tikz를 사용하여 그림을 그리는 방법에 대해 알아보겠다.
\end{frame}

\section{들어가기 전에..}
\begin{frame}[fragile]{패키지}
	tikz를 사용하기 위해서는
	\begin{lstlisting}
\usepackage{tikz}
	\end{lstlisting}
	로 tikz 패키지를 추가해야 한다
	\vfill
	\begin{lstlisting}
\usetikzlibrary{}
	\end{lstlisting}
	로 라이브러리를 추가할 수도 있다
	
	유용한 라이브러리로 tkz-euclide 등이 있다
\end{frame}

\begin{frame}[fragile]{사용 환경}
	tikz로 그림을 그리는 방법은 다음 두 가지가 있다
	\vfill
	\begin{lstlisting}
\begin{tikzpicture}[option]
	(tikz command)
\end{tikzpicture}
	\end{lstlisting}
	\vfill
	\begin{lstlisting}
\tikz[option]{(tikz command)}
	\end{lstlisting}
	\vfill
	tikz 명령어 맨 뒤에는 반드시 세미콜론(;)을 붙여야 한다
\end{frame}

\begin{frame}{위치}
	tikz는 데카르트 좌표계와 극좌표계를 사용한다
	\vfill
	오른쪽으로 가면 x좌표가 증가하고

	위쪽으로 가면 y좌표가 증가한다
	\vfill
	화면의 중심은 (0,0)이 아니고 내가 그린 그림이 알맞은 위치에 오도록 알아서 맞춰지므로 점들의 간격만 신경쓰면 된다
\end{frame}

\AtBeginSection[]{}

\section{점}
\MyBeginSection{크기가 없고 위치만 있는 도형}

\begin{frame}{\secname}{표기}
	점은 \((x,y)\) 또는 \((\theta:r)\) 로 나타낸다
	
	\vfill
	(10cm,2pt)같이 길이의 단위를 쓸 수 있다
	
	단위를 쓰지 않으면 기본으로 cm가 적용된다
\end{frame}

\begin{frame}[fragile]{\secname}{node}
	다음과 같이 좌표를 지정할 수 있다
	\begin{lstlisting}
\node[option](name) at (coordinate){text};
\coordinate[option](name) at (coordinate);
	\end{lstlisting}
	\vfill
	점은 화면에 나타나지 않는다
	
	나타나는 것은 text 뿐이다
	\vfill
	점의 이름은 생략할 수 있다
	
\end{frame}

\begin{frame}[fragile]{\secname}{path}
	path를 사용해 여러개의 점을 한꺼번에 지정할 수 있다
	
	\begin{lstlisting}
\path (coordinate) node[options](name){text}
 ... (coordinate) node[options](name){text};
	\end{lstlisting}
	\vfill
	원하는 만큼 node나 coordinate로 점을 지정하면 된다

\end{frame}

\begin{frame}[fragile]{\secname}{example}
	\begin{lstlisting}
\node (a) at (0:3){This is a};
\coordinate (b) at (0,10);
\path (0,0) node(x) {X}
(3,3) node(y) {Y};
	\end{lstlisting}
	
\begin{center}
	\begin{tikzpicture}
	\node (a) at (0:3){This is A};
	\coordinate (b) at (0,3);
	\path (0,0) node(x) {X}
		  (3,3) node(y) {Y};
	\end{tikzpicture}
\end{center}

\end{frame}


\section{선}
\MyBeginSection{임의의 점과 다른 한 점을 연결하는 직선은 단 하나뿐이다}
\subsection{선분}
\begin{frame}[fragile]{\secname}{\subsecname}
	다음과 같이 여러 선분을 이어서 그릴 수 있다
	\begin{lstlisting}
\draw[option] (0,0) -- (1,1) -- (3,0);
	\end{lstlisting}

	\tikz{\draw[thick] (0,0) -- (1,1) -- (3,0);}

	\vfill
	앞에서 점의 이름을 지어주었다면 좌표 대신 이름을 쓸 수도 있다
\end{frame}
\begin{frame}[fragile]{\secname}{\subsecname}
	두 점을 그냥 잇지 않고 택시거리로 연결할 수도 있다
	\begin{lstlisting}
\draw (0,0) |- (1,1);
\draw (2,0) -| (3,1);
	\end{lstlisting}
	
	\tikz{\draw[thick] (0,0) |- (1,1);\draw[thick] (2,0) -| (3,1);}

\vfill
	이런 것도 가능하다
	\begin{lstlisting}
\draw (0,0 |- 1,1) -- (2,0 -| 3,1);
	\end{lstlisting}
	\tikz{\draw[thick] (0,0 |- 1,1) -- (2,0 -| 3,1);}
	
\end{frame}

\subsection{곡선}
\begin{frame}[fragile]{\secname}{\subsecname}
	베지에 곡선\footnote{\url{https://en.wikipedia.org/wiki/B\%C3\%A9zier_curve}\\}을 다음과 같이 만들 수 있다
	\begin{lstlisting}
\draw (0,0) .. controls (1,1) .. (3,0); 
\draw (4,0) .. controls (5,1) and (6,1) .. (7,0);
	\end{lstlisting}
	\tikz{\draw (0,0) .. controls (1,1) .. (3,0);\draw (4,0) .. controls (5,1) and (6,1) .. (7,0);\draw[dashed] (0,0) -- (1,1) -- (3,0);\draw[dashed] (4,0) -- (5,1) -- (6,1) -- (7,0);}
	
	\vfill
	{\footnotesize (그냥 점들을 이은 선분들을 뒤에 점선으로 표시해 보았다)}
\end{frame}

\subsection{화살표}
\begin{frame}[fragile]{\secname}{\subsecname}
	화살표를 만드는 것은 매우 간단하다
	\begin{lstlisting}
\draw[->,thick] (0,0) -- (1,1) -- (3,0);
	\end{lstlisting}
	
	\tikz{\draw[thick,->] (0,0) -- (1,1) -- (3,0);}
	
	\begin{lstlisting}
\draw[<->,dashed] (0,0) .. controls (1,1) .. (3,0);
	\end{lstlisting}
	
	\tikz{\draw[<->,dashed] (0,0) .. controls (1,1) .. (3,0);}

\end{frame}

\section{면}
\subsection{다각형}
\subsection{원}

\section{좌표 평면}
\subsection{좌표 계}
\subsection{함수}

\section{Extra Tips}


\end{document}